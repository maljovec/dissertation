%%% -*-LaTeX-*-

\chapter{Introduction}

From the cosmic scale of astronomy to the atomic scale of chemistry, and from speculative economics to precision engineering, experts in a variety of fields are interested in gathering data in order to better understand the functional relationships between target quantities of interest and their presumed drivers.
%
For instance, what chemical or structural characteristics of a nuclear fuel optimize its safety and efficiency and to what degree and under what conditions?
%
Or to take an example from a different field, what socio-economic factors play the largest role in determining quality of life?
%
For abstract notions such as ``safety'' and ``quality of life'' that can be encoded numerically and measured or simulated, the scientific community has been designing analysis tasks capable of answering these types of questions.
%
Such tasks include identifying paramaters of interest from a field of candidates, efficient collection of data relating the candidate parameters to the outputs, identifying and summarizing trends in the functional relationship of the input space and output space, and building efficient and reliable predictive models.

However, many new and yet unexplored ideas lie at the intersection of fields of study such as statistics, geometry, and visualization.
%
As such, it is important to apply validated ideas from these disparate fields in order to augment and enhance solutions to problems occuring in other areas.
%
By combining complementary techniques, we can gain new understanding about our observed data and even inform the data gathering process.
%
One such field of recent interest is topological data analysis (TDA).
%
TDA is a growing field arising from the combination of computational geometry, topology, set theory, and visualization.

In this work, I investigate the efficacy of topological data analysis (TDA) techniques to problems arising in the realm of uncertainty quantification and risk monitoring.
%
In particular, I place a special emphasis on applications relevant to nuclear safety, such as adaptive sampling, sensitivity analysis, regression, limit surface extraction, and visualization of risk-informed datasets.
%
\textbf{By approximating topological constructs from potentially sparsely
sampled data, we are able to extract and analyze the structure of scientific simulation data stemming from low (one to three) to moderate (on the order of ten) dimensional input spaces to provide actionable insight for domain
scientists.}

This introductory chapter will qualify what kind of data this work applies to and will introduce the specific problem domain areas.
%
Chapter~\ref{ch:definitions} will normalize our language by synthesizing the relevant information from several fields including: data analysis, set theory, algebra, topology, and graph theory.
%
Chapter~\ref{ch:related} frames the contributions of this work in the context of related research and the current state of the art.
%
Chapter~\ref{ch:theory} addresses the theoretical aspects of the work involved by giving a thorough background on the topological constructs used throughout this body of research.
%
Chapters~\ref{ch:adaptiveSampling} and~\ref{ch:visualization} elaborate on novel solutions to problems in adaptive sampling and analysis of generated data.

\section{Problems}
\subsection{Data Generation}
\subsection{Data Visualization}
\subsection{Data Analysis}
