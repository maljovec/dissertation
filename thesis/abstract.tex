%%% -*-LaTeX-*-
%%% This is the abstract for the thesis.
%%% It is included in the top-level LaTeX file with
%%%
%%%    \preface    {abstract} {Abstract}
%%%
%%% The first argument is the basename of this file, and the
%%% second is the title for this page, which is thus not
%%% included here.
%%%
%%% The text of this file should be about 350 words or less.

% Complete in less than 350 words.
On the forefront of research in most scientific fields, experts rely on experimentation and simulation to better understand phenomena occurring in nature and society.
%
The goal of such studies is typically to understand the effect of one or more stimuli on the condition under study.
%
The number of effective ``stimuli,'' or inputs, can vary from a single parameter to hundreds of parameters in modern day experiments.
%
It is therefore important to identify the most crucial ones and to also understand the behavior of the input space in an efficient and reliable manner as obtaining data under such circumstances can be expensive, dangerous, difficult, and/or time-consuming.
%
To this end, researchers have developed effective sampling strategies, uncertainty quantification techniques, regression models, and sensitivity analyses rooted in foundations of the fields of probability, statistics, and geometry.
%
All of these methods are aimed at providing the most knowledge from the least amount of information under the given constraints.

In this work, we explore injecting foundations from the fields of topology and visualization into each of these processes to improve the scientist's intuition and understanding of the data being collected.
%
Specifically, this document focuses on methods that treat multidimensional, numeric data as a scalar function, that is a fixed, arbitrary number of inputs describing a single output value of interest.
%
The term multidimensional is used to convey input spaces of dimensionality larger than the traditional two or three dimensions typically used in visualization.
%
In this way, we are able to synthesize ideas from computational geometry/topology, regression analysis, data mining, and visualization in order to facilitate low-level tasks such as identifying parameters or regions of interest within the input space, choosing informative samples, summarizing trends in the data, and validating models with ground truth datasets.
%
The techniques provided are demonstrated on applications arising from the field of nuclear engineering.

